% Faz com que o ínicio do capítulo sempre seja uma página ímpar
%\cleardoublepage
% Inclui o cabeçalho definido no meta.tex
%\pagestyle{fancy}
% Números das páginas em arábicos
%\pagenumbering{arabic}

\chapter{Introdução}\label{intro}
%\section{Incluindo citações}\label{intro:historico}

As avaliações de aprendizado são fundamentais para o ensino ao apontar o desempenho da turma diante dos conteúdos. Com aplicações frequentes, as atividades permitem ao professor interagir com os alunos e com os materiais pedagógicos para reformulação e aperfeiçoamento da sua metodologia. Desse modo, é com o acompanhamento da disciplina e o apoio ao educando que as atividades formativas permitem a reformulação do processo de ensino-aprendizagem \cite{barreira2006}. Por meio das atividades podemos identificar o domínio dos estudantes sobre o contexto e sua capacidade de realizar inferências sobre o assunto. O papel da avaliação, portanto, é diagnosticar, apreciar e verificar a proficiência dos alunos para que o professor atue no processo de formação de modo a consolidar o aprendizado \cite{oliveira2005}.

Ao observar problemas no modelo de ensino-aprendizagem, o professor pode agir de forma à contorná-los, personalizando a estrutura curricular. É através das atividades, portanto, que  observamos paralelos do conhecimento individual dos alunos. Dessa forma, a mediação tecnológica para a criação, avaliação, recomendação e visualização em dados educacionais visa apoiar o professor na melhoria e no acompanhamento do currículo do aluno \cite{paiva2012}. Com ferramentas de apoio, o tutor pode verificar a aptidão dos estudantes, de forma individual ou coletiva, para adequá-los à disciplina.

\begin{comment}
Nesses casos é essencial que ocorram ajustes na formação dos alunos e nas práticas docentes.

, há o risco de afetar a produtividade discente na disciplina atual e posteriores afetando características essenciais para  senso crítico


O aprendizado observado por cada atividade pode estimular diferentes atos dos alunos dando ênfaze em determinados pontos do conteúdo e sua prática

Observamos com frequência o desenvolvimento da escrita, criatividade, comunicação e trabalho em equipe dos alunos com divisão em equipes, exercício da escrita, a busca textual, os grupos de discussão, dentre outros.

Enquanto o foco de todas as contribuições para organização em sala de aula são centralizados no professor e este lida com várias salas há sobrecarga deste profissional. 
\end{comment}

\section{Contexto}\label{contexto}

A aplicação das atividades visa orientar o professor a agir de forma corretiva na disciplina para administrar o conteúdo assimilado pelos alunos. Segundo \citeonline{bezerra2008}, as verificações de aprendizado podem ser divididas em três modelos para diferentes fundamentos do conhecimento: atividades de identificação, de inferência e de opinião. As questões de identificação requisitam do aluno a localização de trechos do conteúdo apresentado sem remontar a pesquisa realizada na criação do conceito. As questões de inferência realizam exatamente esse laço entre a tarefa a ser realizada e a disciplina, onde o leitor deve associar o conhecimento adquirido em aula na compreensão do contexto apresentado. Por último, as atividades de opinião dependem do contato do avaliado para caracterizar e julgar partes chave do conteúdo, definindo pontualmente sua interpretação das informações. 

Cada um dos métodos avaliativos apresentados nas atividades auxilia o professor na verificação da cobertura do conhecimento e do exercício cognitivo. Diante das tarefas exige-se do aluno uma perícia numérica, motora, criativa, linguística ou didática para representar sua aptidão com o conhecimento. Dessa forma, as atividades buscam agir para desenvolvimento e avaliação das habilidades do discente com o conteúdo aplicado. 

Dentre as atividades para avaliação do aprendizado podemos destacar as questões discursivas pois se enquadram nos três modelos de verificação citados. Esse tipo de questão têm papel especial por desenvolver habilidades essenciais nos alunos enquanto formulam ideias, interpretam o problema, constroem o texto e põe em prática a leitura. É por esse formato de atividade que diversos esforços são criados para incentivo de práticas para leitura e escrita, como os descritos periodicamente no PNLL\footnote{Plano Nacional do Livro e Leitura / Ministério da Cultura - Ministério da Educação - PNLL/MinC-MEC Disponível em: \url{http://www.cultura.gov.br/pnll}}. Porém, esse tipo de atividade demanda muito tempo de análise do professor.

\section{Motivação}

A análise textual consome muito tempo do professor sobrecarregando-o. A falta de tempo para correção de atividades reduz a qualidade da docência pois representa um dos principais indicadores de estresse \cite{assuncao2009}. Enquanto isso, aos alunos, a falta de prática com as atividades causa defasagem nas habilidades básicas, tal como a leitura e escrita. A esse déficit, é atribuído  graves problemas de aprendizagem e baixo rendimento acadêmico \cite{martens2016}. Por isso, é essencial evitar que o tempo do docente seja limitado pelo controle dos métodos de avaliação de aprendizagem, ampliando os planejamentos de conteúdo, aperfeiçoamentos profissionais e a gerência das necessidades dos alunos. Há um empenho portanto para retificar o tempo docente, ou seja, permitir que o tempo reservado para a atividade profissional seja distribuído para as práticas que tenham como foco o aprendizado dos estudantes.

Um método de retificar o tempo docente e contribuir para análise de aprendizado é o emprego de novas tecnologias de apoio ao tutor e a modelagem dos dados educacionais. A tecnologia permite apoiar o desenvolvimento e a modelagem das salas de aula por meio de ambientes virtuais que controlam os materiais e permitem elucidar o professor sobre a desenvolvimento educacional. Conforme \citeonline{lima2006}, a introdução de sistemas computacionais como recurso pedagógico torna o processo de ensino e de aprendizagem mais efetivo, incentivando a interação dos alunos com a disciplina.

Além da modelagem, os métodos de apoio ao tutor podem suportar o processo de avaliação. Esses coletam informações das atividades para compreender as formas de avaliação do especialista, interpretar os padrões da questão e gerar sua própria avaliação, tornando o professor seu supervisor. Para isso, assim como o proposto, vários sistemas buscam redução do esforço do tutor. O algoritmo desse trabalho, então, coleta padrões de atividade para compreender o modelo avaliativo do professor, aplicando-o no processo avaliativo.

\begin{comment}
Este tipo de sistema já é formalmente conhecido com bons desempenhos na verificação de aprendizado de programação \cite{marcia1}\cite{marcia2}. Porém, no caso da verificação das questões discursivas, os diferentes interpretações e formas de escrita, frequentemente sutis, entre as avaliações do docente ampliam a complexidade para suporte à avaliação.


Deste modo, o foco deste trabalho é apoiar a verificação de questões discursivas com seu amplo uso didático e acréscimo educacional. 
\end{comment}

\begin{comment}
Este capítulo apresenta o contexto, motivação e objetivos deste trabalho, bem como os métodos de pesquisa aplicados e sua organização textual, com o intuito de proporcionar ao leitor melhor compreensão do universo no qual se encontra o problema que buscamos solucionar, classificar grandes massas de dados provenientes de redes sociais com menor esforço humano possível nesse processo. 
\end{comment}

\section{Objetivos}

O objetivo desse trabalho é apresentar um modelo de seleção de características que melhore os resultados de sistemas de avaliação de questões discursivas. O modelo utilizado auxilia o especialista com a redução de esforço de correção e identificação dos termos relevantes para compreensão dos documentos de resposta. Por isso, o sistema foi comparado antes e depois do método de seleção com técnicas conhecidas de classificação na tentativa de formar padrões para a atividade.

Os estudos realizados nessa dissertação, em certos aspectos, são uma continuação do trabalho desenvolvido por \cite{pissinati2014-master} com o acréscimo de algumas melhorias. Esse trabalho visa identificar informações necessárias do especialista de forma a melhorar a densidade dos grupos de documentos e sua classificação. Tal objetivo geral pode ser elencado como os seguintes objetivos específicos:

\begin{itemize}
	\item Reproduzir os experimentos realizados por \cite{pissinati2014-master};
	\item Selecionar os itens necessários que aproximem o modelo à correção do especialista;
	\item Desenvolver sistemas classificadores para comparação da efetividade da seleção de características;
	\item Verificar a validade da proposta conforme os resultados da avaliação.
	\item Discutir os resultados obtidos apresentando a eficácia dos modelos;
	\item Gerar um modelo de \textit{feedback} que apresente aos usuários os modelos criados.
\end{itemize}

\section{Metodologia do Trabalho}

O primeiro passo da metodologia foi realizar a revisão da literatura com o objetivo de identificar o histórico da predição de nota, estudos recentes da classificação desses dados educacionais e os formatos de visualização de resultados apresentados pela comunidade de pesquisa.

Com essa revisão foram identificadas as técnicas comuns para extração da informações e a predição de notas. Definindo as ferramentas e os métodos podemos verificar as representações de informação associadas a esses modelos preditivos. Verificado o estado da arte da pesquisa tanto em avaliação assistida pela computação e visualização da informação desenvolvemos os algoritmos de predição de nota para questões discursivas curtas.

O sistema foi testado em mais de 50 bases de dados da Universidade Federal do Espírito Santo - UFES, além de bases de dados utilizadas por algumas referências. A maioria dessas, enviadas através de um Ambiente Virtual de Aprendizagem - AVA Moodle\footnote{Modular Object-Oriented Dynamic Learning - Moodle. Disponível em: https://moodle.net/} e coletadas por um sistema de transferência de informações.

Após os testes nessas bases de dados, por demonstrar-se efetivo na classificação, foi desenvolvido o formato de visualização de informações pelo mapa de características. Os mapas apresentam a forma de identificação do conjunto de respostas utilizada pelo sistema e caracterizam o padrão de avaliação por nota atribuída pelo professor.


\section{Estrutura do Trabalho}

Nesse Capítulo são apresentados os conteúdos trabalhados nessa dissertação, descrevendo suas motivações, suas metodologias e seus problemas. Além dessa Introdução, também compõe esse trabalho os seguintes capítulos:

\begin{itemize}
\itemsep=-.3mm
\item{\textbf{Capítulo \ref{cap2} (Trabalhos Relacionados)}: Apresenta trabalhos semelhantes no âmbito da correção automática e do apoio à avaliação do professor.
}
\item{ \textbf{Capítulo \ref{cap3} (Avaliação Semiautomática de Questões Discursivas)}: Define a estrutura do sistema desenvolvido para apoio à verificação de aprendizagem.  
}
\item{ \textbf{Capítulo \ref{cap4} (Mapa de Características)}: Apresenta o modelo de identificação de informações relevantes para seleção de características e suporte ao método de classificação e visualização de resultados. 
}
\item{ \textbf{Capítulo \ref{cap5} (Experimentos e Resultados)}: Detalha os experimentos realizados, mostrando os resultados obtidos e os desafios encontrados.
}
\item{ \textbf{Capítulo \ref{conclusao} (Conclusões e Trabalhos Futuros)}: Discute os resultados obtidos e as ações posteriores disponíveis para elaborar ainda mais as contribuições desse trabalho.
}
\end{itemize}

